% 1
\section{序論}

% 1.1
\subsection{背景}
我々は、PCを操作する際にマウス、キーボードを使用する。しかしながら、これらのインターフェースを使いこなすのには時間が必要であり、直感的に扱えるインターフェースとは決して言うことはできない。この問題を解決する一例として、ハンドジェスチャーが挙げられる。この技術により、簡単な操作であれば直感的に行うことができるようになり、より多くの人々が扱うことができる。その一方で、ジェスチャーの種類には限りがあるため、操作が複雑になればなるほど、それを純粋なジェスチャーのみで表現することは難しくなってしまう。そこで、近年ではVR技術と組み合わせることにより、この欠点を補う試みが多々見られる。これにより、多くのシチュエーションに対して、ジェスチャーによる操作を応用することが可能になるため、今後、ハンドジェスチャーはより生活に密接した存在になることが予想される。だが、この進歩の途上には一つの新しい問題が存在する。それは、カメラ等の既存のセンサーが使用できない場所での、ジェスチャーの使用が求められる、というものである。具体例を挙げると、トイレのような倫理的にカメラの使用が憚られる場所がある。この場所では、非接触型インターフェースとして、ハンドジェスチャーの需要が高まると予想されるにも関わらず、カメラをセンサーとして使用した既存の技術は使用することができない。そのような状況への対策として、カメラ以外のセンサーを用いたハンドジェスチャー技術が求められている。

% 1.2
\subsection{目的}
主に二つのことを主眼を置いている。一つ目は、ハンドジェスチャーとVR技術を組み合わせることによって、直感的な操作を行うことができるインターフェースの提案をする、ということである。具体的には、3D空間上に作成したオブジェクトに触れられるようにすることを目指す。二つ目は、ジェスチャーを認識するセンサーとして、カメラを使用しないことにより、これまで利用できなかったシチュエーションでもジェスチャーを利用可能にする、ということである。

% 1.3
\subsection{構成}
本論文は、全六章から構成される。第二章では、類似研究を紹介する。第三章では、研究の概要と提案手法を記述する。第四章では、提案したシステムの具体的なアルゴリズムを説明する。第五章では、システムを動作させた結果を提示する。最後に、第七章では、これらのことを踏まえて、考察、結論等のまとめを行う。