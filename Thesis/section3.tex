% 3
\section{研究概要}

% 3.1
\subsection{概要}
ハンドジェスチャーを使用する際に用いられるセンサーとして、一般的とされているWebカメラではなく、レーザー光を用いる。このことは、従来ジェスチャーを使用できなかったシチュエーションや場所においても、使用することを可能とする。

具体的なレーザー光の使用方法を説明する。多数のレーザーと、それと同数の受光器を用意して、お互いが対面するように配置することにより、その間を手が遮った際に検出できるようにする。この検出した情報から手のおおよその形状を認識する。予め3Dオブジェクトを配置した、3Dの仮想空間を用意して、そこに取得した手の形状を描画し、3Dオブジェクトを動かすことができるようにする。このことで、レーザー光を用いたデバイスがインターフェースとして使用することができることを証明する。

本研究では、これを2台のWebカメラを用いて作ったシミュレータデバイスを用意して実験を行う。

% 3.2
\subsection{提案手法}
\subsubsection{レーザー光デバイス}
レーザー光を用いたハンドジェスチャーを可能とするため、図1のようなデバイスを作成した。

デバイスの上部に、計24個のレーザーを設置する。それと対になるように、下部に同数の受光器を設置する。この間を手が通過すると、レーザー光が遮られたことを受光器が検出する。実際に作成したデバイスを用いて、取得した手情報を画像に描画したものを、図2に示す。図3は使用したデバイスの写真である。

図2より、レーザー光を用いて手の形状を認識することが可能であることが伺える。しかしながら、このデバイスには問題が存在する。それは、レーザー数が足りないため、手の情報を得るためには、スキャナーのように手を固定した形で通過させる必要がある。このことは動的な手情報を得ることができないことを示している。つまり、VRと組み合わせた直感的なハンドジェスチャーインターフェースを作ることは、不可能である。だが、図2の結果から、手の範囲より広くレーザーを設置することが出来れば、動的に手情報を得ることができると推測される。% このデバイスのイメージ用意してもいいかも
このことを踏まえて、レーザー光の代わりにWebカメラを用いたシミュレータデバイスを作成した。

\subsubsection{Webカメラを用いたシミュレータデバイス}