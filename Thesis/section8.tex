% 7
\section{総括}

% 7.1
\subsection{システムについて}
本研究での目的を改めて述べる。一つは、ハンドジェスチャーとVR技術を組み合わせることによって、直感的な操作を行うことができるインターフェースの作成である。二つ目は、カメラを使用しないことにより、これまで利用できなかったシチュエーションでもジェスチャーを利用可能にする、ということである。

二つ目の、カメラを使用しないジェスチャーの利用についてだが、十分に要件を満たしていると考えている。今回使用したシミュレータデバイスだが、実際の光レーザーを使用したデバイスのデータと、ほぼ同じ形でデータを取得することができる。すなわち、4章の図4-3の格子点と同じ数だけのレーザー、および受光器を用意することができれば、Webカメラによるシミュレータでなくとも、今回の実験と同様の結果が得られることが推察できる。しかしながら、必要となるレーザー光を計算すると、縦、横それぞれにおいて、$32 \times 24$ 個を使用することになり、合計で1536個のレーザー光を使用することになる。これを一個人で用意することは難しいため、実際にデバイスを作成しようとすると、一企業単位の組織が必要となる。

一つ目の直観的な操作ができるインターフェースについてだが、今回実現できたとは言うことはできない。その原因だが、内部外部判定を行う際の、計算速度が問題としてあげられる。本システムでは、オブジェクトと手が重なった際に、オブジェクトを移動させて、再度内部外部判定を行っている。この内部外部判定の計算に時間がかかるため、あまりにも早く手を動かすと、計算が追い付かなくなり手がオブジェクトに食い込んでしまう。これにより、オブジェクトの挙動が想定外のものとなってしまう。これを解決するためには、より早い内部外部判定アルゴリズムを導入する必要があると考えられる。

% 7.2
\subsection{実験結果について}
概ね想定通りの結果を示したが、図6-3、6-4での内部外部の誤判定は想定外のものとなった。同様のアルゴリズムを用いた、第5章の図5-13では、実験で使用した図形より複雑なものにも関わらず、内部外部の誤判定は起こらなかったからである。この誤判定は図形の境界線付近で起こっているため、改良型の外積計算によって生じている。おそらく、入力される図形が第5章の図5-13のようにある程度複雑な図形の場合は、このような誤判定は起こらないが、今回の実験のように、あまりにもシンプルすぎる図形の場合、誤判定が起こるのと予想される。本研究の最終目標はVR技術との組み合わせで、インターフェースを作ることである。従って、三点のみで構成されている、あまりにもシンプルすぎる図形を想定していない。ある程度複雑な図形を想定しているため、全て第4章、4.3.3 の計算を適用している。そのことが、今回の誤判定につながったと考えられる。


% 7.3
\subsection{発展}
三次元オブジェクトの内部外部判定を行う際に、層毎に分けることで二次元の内部外部判定を行っている。この手法でも、十分な判定結果を得られるが、三次元を三次元のまま考えて内部外部判定を行う手法を用いると、同様かそれ以上の結果を得られると考えられる。従って、より正確な結果を得るためにも、今後はそのような手法について模索するべきだと考えられる。しかしながら、計算量が大幅に増えることは確実であり、これを解決するための手段も同時に考えなければならない。
